\begin{lstlisting}
##########################################################
##########################################################
###
### Welcome, one and all, to the world's first and only
###
###              Exceptionally Amazing
###           Home-Equity-Position-Market
###            Prognosticating Simulator!
###
### Brought to you by Daniel Baron and Emily Searle-White.
###
##########################################################
##########################################################   (#tag)


############################################
## Simulation settings
############################################
# to do: make output nice and easy. Interactive?
output= 'trackOne'
fieldNames = [
    'Time',
    'Home Value',
    # 'Waffles',    # If you don't need a particular item,
                    # just comment out the line -- DON'T DELETE!
    'Equity',
    'HEP Expectation'#,
    #'Investor Valuations',
    #'E.D.B.',       # Expectation-Derived-Bid; the correct auction
                    # bid if you only care about the expected value
                    # of the HEP.
    #'Investor Bids',
    #'Reserve',
    #'Best Bid',
    #'Winning Bidder' #, # Make sure there is no comma after the
                        # final entry in the list.
    ]

realizations = 100

# All time quantities are in years
duration = 5
# We want to adjust the time resolution within each simulation
# run to improve efficiency.
from fractions import Fraction
longTimeStep = Fraction(1,12)   # 30-day months and 360-day
shortTimeStep = Fraction(1,360) # years, for simplicity.



agingRate = 0.02 # annual depreciation due to aging

sim = True
HomeSalesOn=False
trackOne=True
stopAfterPrimary=False
stopAfterSecondary=False
numSources = 1
numInvestors=10
sellHep = 0#######  ########  ####

###########
#Neighborhood settings:

nbhdDict = dict(city = 'St Louis', # Geographic location
          AppreciationMean = 5, # Quoted in percentage points,
            # so that AppreciationMean = 5 implies 5% annualized
            # home appreciation.
          sdMultiplier = 0.8, # The geometric std deviation of
            # home appreciation is derived from AppreciationMean
            # and this multiplier, so that AppreciationMean = 5
            # and sdGeoMultiplier = 0.8 imply that a
            # (1 + 0.8) * 5% = 8%
            # return is one standard deviation above the mean.                
          medianPrice = 2.257 * 10**5, # Median home price
          percent99= 3, # 99th percentile of home prices,
            # expressed as a multiple of the median.
          pHIP= .52, #Percentage of households reporting at
            # least one Home improvement project in 2 years.
          pDisaster= NotImplemented, # Percentage of households
            # reporting at least one disaster repair project in
            # the 2 year reporting period.
          initialHomes = 3, # Homes seeking HEPs at t=0.
          spawnRate = 3 # New homes seeking HEPs per year
          )
if stopAfterPrimary or stopAfterSecondary or trackOne:
    nbhdDict['spawnRate']=0
    nbhdDict['initialHomes']=1
nbhdDicts = [nbhdDict]

auctionLength = 5*shortTimeStep #Controls when auctions stop.
    # Not yet implemented.


############################################
## Imports, Constants, and Utilities
############################################

from random import random, gauss, choice, \
     shuffle, lognormvariate, expovariate
from math import floor, ceil, log, exp, modf
from types import InstanceType
from csv import DictWriter as dictWriter, DictReader as dictReader
from os import chdir, mkdir
err = RuntimeError

gauss99 = 2.326347874 # 99th percentile
sellDateLambda = 1./13 * log(2) # Parameter of the exponential
  # distribution governing homeowner turnover.

from time import ctime as __now__
now = lambda: __now__().replace(':','..')

class Q(Fraction):
# Modification of Python's rational number class:
# automatically limits the denominator to reasonable values,
# and stringifies in mixed-number form for readability.
    def __new__(cls, numerator=0,denominator=None):
        X=Fraction(numerator,denominator)\
           .limit_denominator(1000)
        return X
    def __str__(self):
        if self.denominator == 1: return str(int(self))
        elif abs(self)<1:
            return str(self.numerator) + '/' \
                   + str(self.denominator)
        else:
            return str(int(self)) + ' ' \
                   + str( abs( self-int(self) ))

class dataClass:
# This is just a generic data wrapper.
    def __init__(self, **args):
        self.__dict__.update(args)

# Keeps track of how many of each object have been made so far.
index = dataClass(last={})

# The discount rate.
riskFree = dataClass(rate=log(1.03));
riskFree.Rate=riskFree.rate

# Auction length, not yet implemented #to do
auctionLength = dataClass(days=auctionLength)

def poissonvariate(lambd):
# Returns a Poisson-distributed random number.
    if lambd<0: raise ValueError, "The poisson parameter must"\
      "be nonnegative."
    lambd=float(lambd)
    k = 0
    p = random()
    
    # P(X = 0) and P(X <=0):
    P= exp(-lambd)
    cumP = P

    while p>cumP:
        k+=1
        P *= lambd/k    #P(X=k)
        cumP += P       #P(X<=k)
    return k

def mean_and_variance(iterable, corrected=False):
# Compute and return the mean and variance of a list
# of numbers.
    sumX = 0
    squareSumX = 0
    for i in iterable:
        sumX += i
        squareSumX += i**2

    try: n= float(len(iterable))
    except TypeError:
        n=0.
        for i in iterable:
            n += 1.
    mean = sumX/n
    
    if corrected: divisor = n-1.
    else: divisor = n
    variance = (squareSumX - n*mean**2)/divisor

    return {'mean':mean,'variance':variance}

##########
# Some logic functions.

__any__=any;__all__=all
def any(*args):
    try: return __any__(*args)
    except TypeError: return __any__(args)
def all(*args):
    try: return __all__(*args)
    except TypeError: return __all__(args)

none = lambda *args: not any(*args)
notAll = lambda *args: not all(*args)

def ifElse(condition, valueIfTrue, valueIfFalse=None):
    if condition:
        return valueIfTrue
    else:
        return valueIfFalse
##########

def randomWalk(mu,sigma,dt):  # For modified Brownian motion
    return gauss(mu*dt,sigma*dt**.5)

##########
# Output data to csv
output=dataClass(   name = output,
                    folder = output + '. ' + now(),
                    data = {},
                    isOn = 1#False
                    )
if output.isOn:
    mkdir(output.folder); chdir(output.folder)
del mkdir, chdir

def investorFields(field):
    if field in fieldNames:
        k = fieldNames.index(field)
        n = k + numInvestors
        j=0
        
        field = field.split(" ")[1][:-1] # assuming the plural just ends in 's'
        field = field + ' #'
        while k < n:
            j += 1; k += 1
            fieldNames.insert(k, field +str(j))

investorFields('Investor Valuations')
investorFields('Investor Bids')    

def makeOutput():
    with open(output.file,'ab') as f:
        writer = dictWriter(f,fieldNames, extrasaction='ignore')
        writer.writerow(output.data)
##        for key in output.data:           db
##            print key,output.data[key]   db

def prep_outFile():
    output.file = 'realization '+str(realization)+'.csv'
    output.data = dict((name,'') for name in fieldNames)
    headers = dict((name,name) for name in fieldNames)
    f=open(output.file, 'wb')
    
    writer = dictWriter(f,fieldNames, extrasaction='ignore')
    writer.writerow(headers)
    f.close()
#########
        
############################################
## Class Declarations
############################################
## Much of the implementation relies on classes and instances
## to represent the abstract "HEPs", "Investors", etc.
## Some of these are little more than data wrappers, but
## there are also big chunks of the simulation procedure coded 
## here as class methods.

class thingClass:
# The parent class; everybody else inherits these attributes
    
    def __str__(self):
    # Gives each Thing a name for it to print, eg "HEP 28".
    # To do: make it easy to print all relevant data,
    # e.g. "HEP 28-- House: 9, Claim: 2.1%, Owner: Bob."
        name = str(self.__class__).split('.')[1].split('lass')[0][:-1]
        name = name + " " + str(self.id)
        return name

    def copy(self):
    # Makes a shallow copy 
        copy = self.__class__(**self.__dict__) # Most have
          # initialization requirements.
        copy.__dict__.update(self.__dict__)
        return copy

    def __init__(self, **args):
    # Track how many Things of this class have been made,
    # give this new Thing an ID number.
        try:
            index.last[self.__class__] += 1
            self.id = index.last[self.__class__]
        except KeyError: # "self" is the first instance of class
            index.last[self.__class__]=self.id=1


class HEPclass(thingClass):
# The Home Equity Positions.

    def __init__(self,house,**args):
        # Each new-minted HEP has a house, but,
        # as yet, no owner, nor fixed percentage.
        thingClass.__init__(self) 
        self.needsPrimary=True
        self.needsSecondary=False
        self.atAuction=False
        self.owner=None
        self.house=house
        self.percentage=args.get(   # Initialize at 1,
            'percentage',           # except for testing
            args.get('p',1))        # purposes.
        house.heps.add(self)

    def getReservePrice(self):
    # The minimum price that a seller will accept at auction.
        if self.owner == None:
            return gauss(1, .05)*valu(self.house)
        else:
            return gauss(.95,.05)*self.owner.getValu(self.house)

    def expire(self):
    # When the underlying house is sold, remove this HEP from
    # the list and pay the investor.
        heps.remove(self)
        self.house.heps.remove(self)
        if self.owner != None:
            self.owner.liquidWealth += payout(self)
            self.owner.heps.remove(self)

    def transfer(self, newOwner, price=0):
    # Transfer the HEP to a new owner, usually as a result
    # of sale at auction.
        
        if self.owner != None:
            self.owner.heps.remove(self)
            self.owner.liquidWealth += price

        newOwner.heps.add(self)
        newOwner.liquidWealth  -= price
        self.owner = newOwner
        
hepClass = HEPclass


class neighborhoodClass(thingClass):
# Neighborhoods.

    def __init__(self, **args):
    # Function __init__ should receive keyword arguments corresponding
    # to the keys in the Neighborhood Settings dictionary discussed in
    # the Simulation Settings section. The easiest way to do this is
    # unpack such a dictionary in the invocation:
    #    neighborhoodInstance = neighborhoodClass(**neighborhoodDict)
        thingClass.__init__(self)

        # mu is the mean log-appreciation, e.g. log(1.05)
        self.mu = log( args["AppreciationMean"]*0.01 + 1 )
        args.pop("AppreciationMean")
        
        # sigma is the square root of the log-variance, given
        # as a multiplier on the mean annualized appreciation.
        self.sigma = log( (1-1/meanGeo)*args["sdMultiplier"]+1)
        args.pop("sdMultiplier")

        self.__dict__.update(args)
        self.appreciation={t:1}
        self.lastTime = t

        # Find the parameters of the Poisson processes
        # governing Home Improvement Projects and Disasters.
        self.lambda_HIP = -log(1-self.pHIP)*0.5
        self.lambda_Disaster = NotImplemented # to do
        #-log(1-self.pDisaster)*0.5

        # Keep track of the houses in this hood with live HEPs:
        self.houses = set()

        # Get the parameters for the log-normal distribution
        # from which we take initial home prices.
        self.priceMu = log(self.medianPrice)
        self.priceSigma = log(self.percent99)/gauss99
    
    def copy(self, t, tabulaRasa = True):
        copy = thingClass.copy(self)
        if tabulaRasa:
            copy.houses=set()
            copy.appreciation={t:1}
        return copy
            
    def appreciate(self, t):
    # Determine the appreciation ratio for the neighborhood
    # median home price since the last time we checked:
    # app(t,s) = app(s)*exp( mu*(t-s) + sigma*sqrt(t-s)*X ),
    # where t>s and X~N(0,1).

        # Have we gone back in time?
        if t< self.lastTime:
            
            #Scrub all traces of the alternate timeline.
            future = True
            time=self.appreciation.keys()
            time.sort()
            time.reverse()
            for Time in time:
                if Time > t : self.appreciation.pop(Time)
                elif future:
                    future = False
                    self.lastTime = Time
                    break                

        # Calculate the appreciation of the median home price since lastTime
        dt = t-self.lastTime
        dApp = exp(randomWalk(self.mu, self.sigma, dt))
        # Record the total appreciation since t=0.
        self.appreciation[t] =  dApp * \
                               self.appreciation[self.lastTime]

        #update all houses in this hood:
        for house in self.houses:
            house.updateValue(dApp, dt)

        self.lastTime = t
        
        #either update all houses whenever neighborhood is updated,
        # or update everybody only once per month,
        # or keep dictionary of appreciation values.

    def getInitialPrice(self):
    # I assume here that home prices are log-normally distributed;
    # thus the median price determines the mean, mu, of the log-prices,
    # and sigma is such that P( X < percent99*median) = 0.99:
    #
    #       self.priceMu = log(self.medianPrice),
    #       self.priceSigma = log(self.percent99)/gauss99. 
        return lognormvariate(              \
            self.priceMu, self.priceSigma)  \
            * self.appreciation[self.lastTime]

    def getInitialMortgage(self):
    # The median current mortgage debt in the US, as a percentage of the
    # value of the home, is 71%; a 100% mortgage is roughly the 80th
    # percentile value. Homeowners seeking HEPs will, in general, have better
    # than average credit, but will also often be assuming a new mortgage
    # on a newly purchased home -- a mortgage not yet payed down.
    #
    # I assume here that these two effects roughly cancel out, so that
    # the cumulative distribution of debt/value ratios is the same in
    # the HEP world as in the world at large; I further assume that the
    # ratios are log-normally distributed.
    #
        log_point71 = -0.342490309      # log(0.71)
        mortgage_sigma = 0.406941146    # -log(0.71)/z80
        return lognormvariate(log_point71, mortgage_sigma)

    def newHouses(self, t):
    # How many houses need new HEPs created?
    # Poisson distributed.
        return poissonvariate(  
            self.spawnRate * (t-self.lastTime)
            )

        
class houseClass(thingClass):#tag
# Houses. Most of the old neighborhood model is reproduced as methods
# of this class.
#
# I could, at present, move a lot of these methods to the neighborhood
# class, since all relevant parameters come from the hood. But I want
# eventually to include "homeowner profiles", e.g. credit history,
# first time homeowner status, moving habits, and it will be useful
# to have the methods here.

    def __init__(self, neighborhood, t, distribution=False):
        self.hood = neighborhood
        neighborhood.houses.add(self)
        self.value = neighborhood.getInitialPrice()
        self.M0 = neighborhood.getInitialMortgage()*self.value

        # Keep track of all HEPs derived from this house, and
        # the sum of their claim percentages.
        self.heps = set()
        self.percentageSum = 0

        # Get the date at which the homeowner will sell the
        # home, and track whether investors know the homeowner
        # is planning to sell.
        self.sellDate = self.getSellDate(t)
        self.knownSell = False
        
        # Approximate the cumulative probability distribution
        # of HEP values in this neighborhood.
        # We usually skip this step.
        if distribution:
            self.dist = HEP_distribution(self)

    def updateValue(self, dApp, dt):
    # The existing value appreciates by dApp,
    # depreciates because of aging.
    # Any discrete changes are then applied.
                
        self.value *= dApp*(1-agingRate)**(dt)
        
        n = self.getNumDiscrete(dt)
        v = self.getValDiscrete(dt, n)
        self.value += v

        #print 'HIPs: ',n
        #print 'HIP vals: ', v
        

    def getNumDiscrete(self,dt):
    # The AHS data includes the number of homeowners who report
    # having had home improvement projects or disaster repairs
    # in the last two years. If p is the proportion of such households,
    # N is the number of HIPs in a household in two years, and we assume
    # that N~Poisson, then
    #       P( N >= 1 ) = p   =>   P( N = 0 )= 1-p   =>   lambda = -log(1-p),
    # where lambda is the Poisson parameter.
        return poissonvariate(dt*self.hood.lambda_HIP)

    def getValDiscrete(self,dt,n):
    # If the number of discrete changes is given by the Poisson r.v.,
    # then their expected value is determined by the expectation of
    # neighborhood appreciation: since each neighborhood is a bunch
    # of houses, we must have
    #       E(House Appreciation) = E(Hood App.) = e^(mu+0.5*sigma^2).
    #
    # We then assume that the values are normally distributed.
    #
    # Eventually I will add disasters as a second type of discrete change.
    
        mean = self.value * \
               agingRate/((1-agingRate)*self.hood.lambda_HIP)
        sd = mean

        # The sum of n RVs, each N(m,sd), is N(n*m, sqrt(n)*sd).
        mean *=  n
        sd   *=  n**.5
        return gauss(mean, sd)

    def getSellDate(self,t):
    # Data from the NAHB indicates that median tenure of a homeowner
    # is about 13 years. The probability of moving in a given year is
    # not constant -- it's something like P(n) = 1+ 4/n -- but I am going
    # to model it as constant, p = 1 - 2^(-1/13), because doing so makes
    # this a Poisson process with lambda = -log(1-p) = 1/13 * log(2).
    #
    # This will make it easier to estimate the expected payoff of a HEP,
    # since the time until a homeowner sells will be exponentially distributed,
    # and the pdf of the expo distribution is really easy to integrate.
        if HomeSalesOn:
            return t + expovariate(sellDateLambda)

        else:
            return duration + 99

    def remove(self):
    # When the homeowner sells the home, all HEPs derived from it expire
    # and we take it out of all the relevant data objects.
        self.hood.houses.remove(self)
        houses.remove(self)
        
        for hep in self.heps.copy():
            hep.expire()
            
        

class bidClass(thingClass):
# A bid: it has a bidder, and a value.
# The value can be in percentage points
# (if the bid is for the primary auction)
# or in dollars/whatevers
# (in the secondary market).
    def __init__(self,bidder,value,**args):
        self.bidder = bidder
        self.value = value
        
class investorClass(thingClass):
# Investors. We use the probability weighting function in
# http://www-personal.umich.edu/~gonzo/papers/shapewf.pdf
# from figure 2 (though the authors of this paper mention
# this function primarily to show why theirs is better).

    def __init__(self,
                 #wParameter,
                 #vParameter1,
                 #vParameter2,
                 #weightVector,
                 #income,
                 #portfolio,
                 **args
                 ):
        thingClass.__init__(self)
        self.commitments = 0
        self.heps=set()

        self.wParameter = args.get('wParameter', self.get_w())
        self.vParameter1 = args.get('vParameter1', self.get_v1())
        self.vParameter2 = args.get('vParameter2', self.get_v2())
        self.portfolio = args.get('portfolio', self.get_portfolio())
        self.income = args.get('income', self.get_income())
        self.weightVector = self.get_weights(args.get('weightVector'))
        self.liquidWealth = self.portfolio.wealth * \
                            0.1*lognormvariate(0, log(2))
        self.lastTime = t
        self.valus={}

    def get_w(self): return 0.4 + 0.6 * random()
    def get_portfolio(self):
        wealth = lognormvariate(log(10**6),log(2))
        mu = riskFree.rate + gauss(0,riskFree.rate/2)
        sigma = mu**2 + min(mu*0.9, 0)

        return dataClass( wealth=wealth, sigma=sigma, mu=mu)
    
    def get_v1(self): return  100*lognormvariate(0,log(2))
    def get_v2(self): return  lognormvariate(0, log(.75))*4./3
    def get_income(self): return self.portfolio.wealth*gauss(0.2, 0.2)
    def get_weights(self, vector):
        if vector == None:
            vector = [random() for i in range(numSources)]

        s = float(sum(vector))
        return [x/s for x in vector]
        
    def update(self,t):
        dt = t- self.lastTime
        self.lastTime = t
        oldWealth = self.portfolio.wealth

        self.portfolio.wealth *= exp(randomWalk(
            self.portfolio.mu, self.portfolio.sigma, dt))
        self.income *= exp(randomWalk(
            log(1.05), log(1.05),dt))
        self.portfolio.wealth += self.income * dt

        if self.portfolio.wealth <= 0:
            self.liquidWealth = 0
        else:
            dWealth = self.portfolio.wealth-oldWealth
            self.liquidWealth += dWealth \
                              * 0.1*lognormvariate(0, log(2))
        

    def getValu(self,house):
    # An investor's "valuation" for a house
    # is the most he/she would pay for a hypothetical
    # 100% HEP claim on that house;
    # thus [ getValu(house) * p% ] is the investor's
    # valuation of a p% HEP claim on that house.

        if (house in self.valus) and \
           (self.valus[house]['t']==self.lastTime):
            return self.valus[house]['valu']
        if self.portfolio.wealth < 0:
            valu = 0

        else:

            dist = getHouseDistribution(house)

            U_valu = U_func(self.wParameter,
                            self.vParameter1 * \
                            self.portfolio.wealth**(-1),
                            self.vParameter2,
                            dist)
            valu=U_valu

        self.valus[house]={'valu':valu , 't':self.lastTime}
        return valu

    def getSellList(self,t):
        # to do: make sure a hep can't be added to the sell list
        # while it is already at auction
        sellList=set()
        for hep in self.heps:
            if not hep.atAuction:
                p=random()
                if p < sellHep**(t-self.lastTime):
                    sellList.add(hep)
        return sellList

    
class auctionClass(thingClass):
# The parent class for both primary and secondary auctions

    def __init__(self,hep,**args):
    # Each auction is for one HEP;
    # we track the day's bids,
    # the previous day's winning bid,
    # the set of investors who do not
    # not wish to bid on the auction,
    # and the time elapsed since opening the auction.
    #
    # We also record the seller's reserve price.
        thingClass.__init__(self)
        self.hep=hep
        self.bids=set()
        self.startTime = t
        self.noBid = set()
        self.veryOld=[]
        
        self.reserve = self.getBidValue(
            self.hep.getReservePrice()  )
        self.oldWinner=bidClass(None, self.reserve)

    def roundIt(self, x):
    # Round a valuation to the nearest discrete bid step.
    # Since primaryClass.discrete is negative, this
    # rounds *up*: the investor won't bid any lower.
    #
    # For secondaries it of course rounds down.

        return floor(float(x)/self.discrete)*self.discrete

    def close(self):
    # When the HEP is sold, close the auction, transfer
    # ownership. If the reserve price wasn't met, then
    # of course no transfer occurs.
        price = self.price()

        winner = self.oldWinner.bidder
        hep = self.hep
        if winner != None:
            hep.transfer(winner, price)

        auctions.remove(self)

##        print "AUCTION REPORT: ", self
##        if winner == None:
##            print hep
##            print "Reserve not met."
##            print 'winner',self.winner.bidder,self.winner.value
##            print 'reserve',self.reserve
##            print 
##            
##        else:
##            print "Sold, after "+\
##                str(int((t-self.startTime)/shortTimeStep)+1)+\
##                " days at auction:"
##            print hep
##            print "Buyer:"
##            print winner
##            if self.__class__ == primaryClass:
##                print "HEP claim percentage:"
##                print self.oldWinner.value
##            else:
##                print "Sale price"
##                print self.oldWinner.value
##                print "HEP claim percentage:"
##                print hep.percentage
##        print
##        #index.auct=self
##        #raise err, ''
        
class primaryClass(auctionClass):
# One primary auction.

    def __init__(self,hep,**args):
        self.type = "Primary"
        auctionClass.__init__(self,hep,**args)

    def isAsGood(self,bid1, bid2):
    # Is bid1 at least as good as bid2?
    # *Lowest* bid wins.
        return bid1.value <= bid2.value
    
    oldPrice = price = lambda self: 10000.
    # The price in dollars of the winning bid; always $10,000
    # for primary auctions

    def close(self):
    # When the HEP is sold, close the auction,
    # transfer ownership,
    # fix the HEP claim percentage.

    # If reserve price wasn't met, the HEP is never created:
        if self.winner.bidder == None:
            self.hep.expire()
            
        else:
            self.hep.percentage = self.oldWinner.value
            self.hep.house.percentageSum += \
                                         self.hep.percentage    
            
        auctionClass.close(self)
        primaries.remove(self)
        if stopAfterPrimary: step.step = duration+1
        if stopAfterSecondary: step.step = 1
        
    def getBidValue(self, valu):
    # The price of a new HEP is fixed at $10,000;
    #   10,000 / valu  =  p% / 100%.
        return self.roundIt(10000./valu)

    discrete = - 0.0001      #percentage points
    # The bids come in discrete steps.
    # In addition to being accurate (auctions don't
    # work if I can raise the bid by 1/10 cent
    # to win it from you), this makes for the interesting
    # possibility of tie bids.
    #
    # This is negative because we want low bids.

class secondaryClass(auctionClass):
# The class of secondary auctions.

    def __init__(self,hep,**args):
        self.type = "Secondary"
        auctionClass.__init__(self,hep,**args)

    price = lambda self: self.winner.value
    oldPrice = lambda self: self.oldWinner.value
    # The price in dollars of the winning bid; always the
    # amount of the winning bid for secondary auctions.

    def isAsGood(self,bid1, bid2):
    # Is bid1 at least as good as bid2?
    # *Highest* bid wins.
        return bid1.value >= bid2.value

    def close(self):
    # When the HEP is sold, close the auction,
    # transfer ownership.
        auctionClass.close(self)
        secondaries.remove(self)

        if stopAfterSecondary: step.step=duration+1
        self.hep.atAuction=False

    def getBidValue(self, valu):
    # valu == what I would pay for a 100% HEP.
    # Thus, getBidValue == p% * valu == what I 
    # pay for a p% HEP.
        return self.roundIt(self.hep.percentage * valu)

    discrete = 10 #dollars


#####################################################
#   the Auction procedures.
#####################################################

# Regarding "bids":
#
# A "bid" in this program and a bid on the Primarq
# marketplace are not quite the same entity.
# Here, bid.value is the *best* that an investor
# would be *willing* to bid; he will not actually bid
# so much unless forced up in a "bidding war" with
# another investor.
#
# The "winner" and "secondbid" attributes of an
# auction instance track the two best "bids" in the
# above sense; then, at the end of the day, winner.value
# is adjusted to reflect the best bid actually placed--
# each investor may have placed several actual bids or
# even none, but the best bid will be as calculated below.
#
# What we are technically modelling is therefore an iterated
# second-price sealed-bid auction, or iterated Vickrey auction
#   ( http://en.wikipedia.org/wiki/Vickrey_auction ).
# This should behave similarly to a standard "English auction"
# in the limit as timestep -> 0: you and I are at a cattle
# auction; we silently decide how high we want to bid;
# you bid 15 cents/lb, I bid 16 cents; with this new
# information, we might each adjust our private maximums.
#
# Even with a timestep as large as one day, I think we ought
# to see some cool results.


def howManyHEPS(house):
# The number of HEPs a homeowner wants to sell;
# i.e., 1/($10,000) * (desired money).

    if stopAfterPrimary or stopAfterSecondary: return 1 #db

    # Guess at the claim percentage of a single HEP.
    p1 = 10000./valu(house)

    # The homeowner always retains at least 50% of the equity.
    if p1 > .5:
        return 0

    else:
        # Uniformly distributed, and at least 1:
        Min = 1
        Max = int(0.5/p1)
        return Min + int( random()*(Max-Min) )
                                

def isFinished(auct):
# Returns a Boolean:
# We call an auction 'finished' when
# nobofy has placed a new bid for a
# while ( a while = 1 day ).
#
# Alternatively, the auction closes if no one has met
# the seller's reserve price in 5 days of bidding.
# 
    B1 = (auct.oldWinner.bidder != None) and \
         (auct.isAsGood(auct.oldWinner, auct.winner) )

    B2 = (t - auct.startTime >= (5-1)*shortTimeStep) and \
         (auct.winner.bidder == None)
    
    return  B1 or B2


def newAuctions():
# Creates new auction instances as needed.
#
# to do:  make a set of guys who need auctions,
# so it won't have to loop through all guys every timestep.
    newPrimaries = set()
    newSecondaries = set()
    anyNewP = False
    anyNewS = False
    for hep in heps:
        if hep.needsPrimary:
            hep.needsPrimary=False
            newPrimaries.add(primaryClass(hep))
            anyNewP = True
        if hep.needsSecondary:
            hep.needsSecondary=False
            hep.atAuction=True
            newSecondaries.add(secondaryClass(hep))
            anyNewS = True

    if anyNewP:
        primaries.extend(newPrimaries)
        auctions.extend(newPrimaries)
        primaries.sort( key = lambda auct: valu(auct.hep.house),
                        reverse = True)
    if anyNewS:
        secondaries.extend(newSecondaries)
        auctions.extend(newSecondaries)
        secondaries.sort( key = lambda auct: valu(auct.hep.house),
                        reverse = True)

    if anyNewP or anyNewS:
        auctions.sort( key = lambda auct: valu(auct.hep.house),
                        reverse = True)


def getBidderCash():
    bidderCash = {}
    for investor in investors:
        cash = [investor.liquidWealth]
        cash.append(cash[0]-investor.commitments)
        bidderCash[investor] = cash

        investor.commitments=[]
    return bidderCash
        


def runAuctions(theAuctions, bidderCash):
# Given a set of auctions and each bidder's available cash, find the
# winning bid for each auction.
    recursionDepth.n+=1#db
    #if recursionDepth.n>1 : print 'depth:',recursionDepth.n,'.  t:', t
    
    toRemove = set()
        
    for auct in theAuctions:

        #initialize some variables
        auct.winner = auct.oldWinner.copy()
        secondbid = auct.winner.copy()
        auct.tie = False
        newWinner = False

        #get bids
        for investor in investors:
            cash = bidderCash[investor][0]

            #House valuation.
            valu = investor.getValu(auct.hep.house)

            # Best willing bid.
            value = auct.getBidValue(valu)
            if investor in auct.noBid:
                value = auct.oldWinner.value
            
                               

            # Does the bidder have enough cash to make this bid?
            if auct.type == "Primary":
                if cash < 10000:
                    auct.noBid.add(investor)                          
            else:
                if cash < value:
                    value = ifElse(investor==auct.oldWinner.bidder,
                                   max(auct.oldWinner.value,cash),
                                   cash)         

            if (investor in auct.noBid) or\
               (investor==auct.hep.owner):
                pass
            else:
                    
                bid = bidClass(investor, value)
                auct.bids.add(bid.copy())

                # Keep "winner", "secondbid" up to date:
                if auct.isAsGood(bid,auct.winner):

                    # Is this bid a change from yesterday's status quo?
                    if any(
                        auct.oldWinner.bidder == None,
                        none(
                            auct.isAsGood(auct.oldWinner, bid),
                            auct.oldWinner.bidder==bid.bidder
                            )
                        ):
                        newWinner = True

                    secondbid = auct.winner.copy()
                    auct.winner = bid.copy()
                    if secondbid.value==auct.winner.value:
                        auct.tie = True
                    else: auct.tie = False

                # The investor willing to bid second-best
                # might be queried *after* the best bidder
                # has been queried:
                elif auct.isAsGood(bid,secondbid):
                    secondbid = bid.copy()

                    # Is this bid a change from yesterday's status quo?
                    if none(
                        auct.oldWinner.bidder == bid.bidder,
                        auct.isAsGood(auct.oldWinner, bid)
                        ):
                            newWinner = True
                    
        # Is there a tie?
        if auct.tie:
            winners = [] # everybody who tied goes here.
            for bid in auct.bids:
                if bid.value == auct.winner.value:
                    winners.append(bid)

            # Pick one at random:
            # this is the lucky one who happened to place the actual
            # bid on the marketplace first. The others were
            # unwilling to outbid him.
            auct.winner = choice(winners)
            
        else:
            # If there was no tie, then the winning bidder might not
            # have gone fully as high (low) as he/she would have been
            # willing, but only a little bit higher (lower) than
            # the second-best bidder was willing to go.
            auct.winner.value = secondbid.value + auct.discrete

        # Each investor object tracks that investor's bid commitments.
        if newWinner:
            auct.winner.bidder.commitments.append(auct)
        else:
            # This auction is finished for the day, and
            # yesterday's winning bidder must stand by his
            # commitment.
            toRemove.add(auct)
            auct.winner = auct.oldWinner
            if auct.oldWinner.bidder != None:
                bidderCash[auct.oldWinner.bidder][0] \
                        -= auct.oldPrice()

        
##        for i in d:
##            if d[i]>100:
##                for inv in investors:
##                    if inv.id==int(i.split('#')[1]):
##                        print i
##                        print inv.getValu(auct.hep.house)
##                        print auct.getBidValue(inv.getValu(auct.hep.house))
##                print d
##                print "TYPE: ", auct.type
##                #raise err, 'Way too high'

        
        if not trackOne:
            d=dict(("Bid #"+str(bid.bidder.id), bid.value)
               for bid in auct.bids)
            output.data.update(d)

        auct.bids = set()

    for auct in toRemove:
        theAuctions.remove(auct)
    
    # Can everyone afford all the commitments they've made?
    for investor in investors:
        skint = False

        # Sort in descending order of happiness.
        investor.commitments.sort(
            key = preference, reverse = True )

        for auct in investor.commitments:
            # Yesterday's best bidder is no longer
            # commited to the bid:
            if auct.oldWinner.bidder != None:
                bidderCash[auct.oldWinner.bidder][1] \
                        += auct.oldPrice()
            
            # Was this the same best bidder as yesterday?
            if (auct.oldWinner.bidder == investor) and skint:
                #Would you rather have spent the money elsewhere?
                # Don't bid
                auct.noBid.add(investor)
                bidderCash[investor][1]-=auct.oldPrice()

##                else:
##                    theAuctions, bidderCash, skint = \
##                            manageCommitments(
##                        theAuctions,
##                        auct,
##                        investor,
##                        bidderCash,
##                        skint,
##                        sameWinner=True
##                                            )
##                    auct.testing #db

            else:
                theAuctions, bidderCash, skint = \
                            manageCommitments(
                        theAuctions,
                        auct,
                        investor,
                        bidderCash,
                        skint
                                            )
                auct.testing #db
                
        #Clear commitments
        investor.commitments = []

    # Base case:
    if len(theAuctions) == 0 or recursionDepth.n>50:#to do
        for investor in investors: investor.commitments = 0
        return

    # Recursive step:
    else:
        return runAuctions(theAuctions, bidderCash)


def manageCommitments(theAuctions,
                      auct,
                      investor,
                      bidderCash,
                      skint):
    # Enough money?
    if auct.price() <= bidderCash[investor][1]:

        # We consider this auction finished for today.
        theAuctions.remove(auct)

        # We need to ensure that the sum of the claim
        # percentages over all HEPs on a house is never
        # greater than 50%.
        if percentagesOK(auct):
            
            # Deduct the price of the HEP from
            # available cash.
            bidderCash[investor][1] -= auct.price()
            bidderCash[investor][0] -= auct.price()
            
        else:

            # Change the reserve price to ensure that
            # the percentages will be OK. #db auct.winner also
            auct.reserve = 0.5 - auct.hep.house.percentageSum
            auct.oldWinner = bidClass(None, auct.reserve)

            # Since this bidder placed the day's lowest
            # bid, and that wasn't low enough, we still
            # leave the auction finished for the day.
    else:
        skint = True
        if auct.oldWinner.bidder != None: #to do: just put None in bidderCash
            bidderCash[auct.oldWinner.bidder][1] -= auct.oldPrice()
            
    auct.testing = 1 #db
    return theAuctions,   bidderCash, skint


def finishAuctions():
# At the end of the day, check whether each auction is over or
# still ongoing. If ongoing, then update the oldWinner bid.
    toRemove=set()
    for auct in auctions:

    
        if not trackOne:
            output.data.update({
                'Home Value':auct.hep.house.value,
                'Equity':auct.hep.house.value-auct.hep.house.M0,
                'HEP Expectation':expectation(auct.hep),
                'E.D.B.':auct.getBidValue(expectation(auct.hep)),
                'Reserve':auct.reserve,
                'Best Bid':auct.winner.value,
                'Winning Bidder':auct.winner.bidder
                })
            output.data.update(
                dict(
                    ("Valuation #"+str(I.id),
                     I.getValu(auct.hep.house)) for I in investors
                ))     

        auct.noBid = set()

        # determine whether to close
        if isFinished(auct):

            #add to the list of auctions to close
            toRemove.add(auct)

        else:

            # End of the day, today's data becomes 'old'.
            auct.oldWinner = auct.winner.copy()
            if auct.winner.bidder!=None:
                auct.winner.bidder.commitments += auct.price()
            auct.veryOld.insert(0,auct.winner.copy())
            if t - auct.startTime >= auctionLength.days:
                 auct.veryOld.pop()

    # Close auctions
    for auct in toRemove:
        hep=auct.hep#db
        hep.house
        auct.close()
        if hep.owner==None and hep in house.heps:
            raise err, 'ghost hep'
        
    # If there are any alive, we need the short time step.
    if len(auctions) > 0:
        step.step = shortTimeStep
    #db print len(auctions),step

    #print "--------------------------------"#db

    return


def percentagesOK(auct):
# We need to ensure that the sum of the HEP claim percentages
# on a house is never greater than 50%.
    if auct.type == 'Primary':
        S = auct.hep.house.percentageSum
        if auct.winner.value > 0.5 - S:
            return False

    return True
        
            
##################################################
#   Investor Valuations
##################################################

def wFunc(p, parameter):
    A= parameter
    return (p**A)/(p**A + (1-p)**A)**(1/A)

def vFunc(x, parameter1, parameter2):
    A,B =parameter1,parameter2
    if x > 0:
        return log(A*x + 1) / A
    else:
        return -log(B*A*-x + 1) / (B*A)

def U_func(wParameter,vParameter1,vParameter2, dist):

    w=wParameter
    v1=vParameter1
    v2=vParameter2

    n = int(1/dist.resolution)
    Sum = 0
        
    for a in range(1,n+1):
        p=float(a)/n
        x = dist(p)
        if x <0:
            Sum += (wFunc(p,w)-wFunc(p-1./n,w))*vFunc(x,v1,v2)
        else:
            Sum += (-wFunc(1-p,w)+wFunc(1-p+1./n,w))*vFunc(x,v1,v2)
    return Sum
        
        
def valu(house):
# The valuation of a house from the point of view
# of the "general public". Just the expectation for now.
    hep = dataClass()
    hep.percentage=1
    hep.house=house
    mu = house.hood.mu - house.hood.sigma
    sigma = 0
    return expectation(hep, mu=mu,sigma=sigma)

def preference(auction):
# Measures how happy an investor is with a purchase; given by
#       p(commitment) = HEP.percentage * valuation / price.
# Notice that this will always be at least 1.

       
    if auction.type == "Primary":
        return auction.winner.value \
             * auction.winner.bidder.getValu(auction.hep.house) \
             / 10000.
    else:
        # For homes "underwater" on their mortgages, this might
        # result in division by zero... fix later.
        return auction.hep.percentage \
             * auction.winner.bidder.getValu(auction.hep.house) \
             / auction.winner.value
    
    
    

def payout(hep, salePrice = None, M0=None):
# The payout of a HEP upon its expiration,
#       P(S) = p(S - M0),
# where S is the final sale price,
# M0 is the initial mortgage amount,
# and p is the HEP claim percentage.
    if salePrice == None:
        return hep.percentage * \
               (hep.house.value - hep.house.M0)
    else:
        return hep.percentage * \
               (salePrice - M0)

def expectation(hep=None, pretty=False,**args):
# Expectation, in today's dollars, of the value of a HEP.
    '''Takes a HEP, or the following: mu, sigma, p, s0, M0.
    Optional: r, lambd.'''
    
    # Get parameters from keyword arguments,
    # or use global parameters.
    r = args.get('r', riskFree.Rate)
    lambd = args.get('lambd', sellDateLambda)

    # If 'hep' is a HEP object:
    if type(hep)==InstanceType:
            mu = hep.house.hood.dist.logMu
            sigma = hep.house.hood.dist.logSigma
            p = hep.percentage
            s0 = hep.house.value
            M0 = hep.house.M0

    # Use parameters from keyword arguments instead of
    # those from the hep, or if there is no hep.
    if 'mu' in args: mu = args['mu']
    if 'sigma' in args: sigma = args['sigma']
    if 'percentage' in args or 'p' in args:
        p = args.get('percentage',args.get('p'))
    if 's0' in args: s0 = args['s0']
    if 'M0' in args: M0 = args['M0']

    
            
    if mu+.5*sigma**2-lambd-r >= 0:
        riskFree.rate = r = mu+.5*sigma**2-lambd+0.0001
        print 'Rate adjustment!'
        
    expectation = p*lambd * \
           (-s0/(mu+.5*sigma**2-lambd-r)  \
            -M0/(lambd+r))

    # Print in readable format: e.g., $426,433.83
    if pretty:
        def pretty(money):
            if money < 1:
                s = str(round(money, 2))
                s += '0'*(4-len(s))
                return s
            
            E = floor(log(money, 1000))
            big = int(money/1000**E)
            
            if E ==0 :
                return str(big) + \
                       pretty( money-floor(money) )[1:]
            else    :
                s = pretty(money-big*1000**E)
                s = '0'*(3-len(s.split(',')[0])) + s
                return str(big) + ',' + s
            
        print "$" + pretty(expectation)

    return expectation


def HEP_distribution(house):
    
    t=0
    hood=house.hood.copy(t) #to do: make the copy function go deep
    hood.houses=set()
    house = houseClass(hood, t, distribution = False)
    
    value= 2250000  
    M0 = 0.8*value  

    def reset():
        t=0
        hood.appreciation={t:1}
        hood.lastTime=0
        house.value= value
        house.M0= M0
        house.sellDate=house.getSellDate()

    reset()
    hep = HEPclass(house);
    hep.percentage = 1

    v0 = expectation(hep)
    v1expect = mu = v0*exp(riskFree.rate)
    vList=[]
    
    real = 1
    n = numReals = 1000.
    while real<=numReals:
        if house.sellDate<=1:
            t=house.sellDate
            hood.appreciate(t)
            v1=(house.value-house.M0)*exp(riskFree.Rate*(1-t))

        else:
            t=1
            hood.appreciate(t)
            v1=expectation(hep)

        vList.append(v1)
        reset()
        real+=1

    mv = mean_and_variance(vList,True)
    mean = mv['mean']
    
    n=numReals; mu = v1expect
    sampleVar = mv['variance']
    Var = (sampleVar*(n-1) + n*mean**2  \
           -2*n*mean*mu + n*mu**2) /n

    def million(x):
        return str(round(x/10.**6,2))+" million."
    
    
    print "Initial Value: ", million(value)
    print "Mortgage:  ", million(M0)
    print
    print "Analytic Expectation: ", million(mu)
    print "Sample Mean: ", million(mean),mean
    print
    print "Variance from Expectation: ", million(Var)
    print "Sample Variance: ", million(sampleVar)
    print
    print "Standard Deviation from Expectation: ",\
          million(Var**.5)
    print "Sample Standard Deviation: ",\
          million(sampleVar**.5)

def getHoodDistribution(hood, t=0, resolution = 0.01):
# Simulates and returns the approximate cumulative
# distribution of the appreciation of a house in one year,
#        S1 / S0,
# given the parameters in the neighborhood.
#
# The distribution is in the form of a python function.
# It interpolates linearly when asked for a value at a
# finer resolution than recorded; e.g.,
# dist(0.90) returns the recorded 90th percentile value,
# but dist(0.903) returns 0.7*dist(0.90)+0.3*dist(0.91),
# assuming that resolution=0.01.

    hood = hood.copy(t)
    #print hood.appreciation

    house = houseClass(hood, t, distribution = False)
    value0=hood.medianPrice
    house.M0 = value0*.71    # Not really important until
                             # we include default risk.

    # When using the chi^2 test to analyse a multinomial
    # distribution, the rule of thumb is to set the sample
    # size at least high enough that the expected number of
    # occurences in each bin is at least 5.
    #
    # We're not actually doing anything with the chi^2 test,
    # but we might in the future. Anyway, I took that rule and
    # and doubled it for good measure, and because computers
    # are fast.
    numData = int(1/resolution)
    multiplier = 10
    reals = numData*multiplier

    listX=range(reals)
    
    # Do a year's appreciation
    for realization in range(reals):
        t+=1
        house.value=value0

        hood.appreciate(t)

        # Record the ratio
        X=house.value/value0
        listX[realization] = X

    return getDistribution(listX, resolution)


def getHoodSimulation(hood, t=0, resolution = 0.01):
# Simulates and returns the approximate cumulative
# distribution of the appreciation of a house until it sells,
#        S(t) / S0,
# given the parameters in the neighborhood. Also attaches the
# list of sell dates.
#
# The distribution is in the form of a python function.
# It interpolates linearly when asked for a value at a
# finer resolution than recorded; e.g.,
# dist(0.90) returns the recorded 90th percentile value,
# but dist(0.903) returns 0.7*dist(0.90)+0.3*dist(0.91),
# assuming that resolution=0.01.

    hood = hood.copy(t)
    #print hood.appreciation

    house = houseClass(hood, t, distribution = False)
    value0=hood.medianPrice
    house.M0 = value0*.71    # Not really important until
                             # we include default risk.

    # When using the chi^2 test to analyse a multinomial
    # distribution, the rule of thumb is to set the sample
    # size at least high enough that the expected number of
    # occurences in each bin is at least 5.
    #
    # We're not actually doing anything with the chi^2 test,
    # but we might in the future. Anyway, I took that rule and
    # and doubled it for good measure, and because computers
    # are fast.
    numData = int(1/resolution)
    multiplier = 100
    reals = numData*multiplier

    listX=range(reals)
    
    # Do a year's appreciation
    for realization in range(reals):
        t=house.sellDate
        house.value=value0

        hood.appreciate(t)

        # Record the ratio and the time.
        X=house.value/value0
        listX[realization] = (X,t)
        t=0
        house.sellDate=house.getSellDate(t)
        hood.appreciation={t:1}
        hood.lastTime=t

    listX = dataClass(resolution=resolution, listX=listX)
    return listX


def getHouseDistribution(house, t=0, resolution = 0.01):
# Simulates and returns the approximate cumulative
# distribution of the appreciation of a house until it sells,
#        S(t) / S0,
# given the parameters in the neighborhood. Also attaches the
# list of sell dates.
#
# The distribution is in the form of a python function.
# It interpolates linearly when asked for a value at a
# finer resolution than recorded; e.g.,
# dist(0.90) returns the recorded 90th percentile value,
# but dist(0.903) returns 0.7*dist(0.90)+0.3*dist(0.91),
# assuming that resolution=0.01.

    try:
        if house.dist.lastTime==t:
            return house.dist
    except AttributeError:
        pass

    s0=house.value
    M0=house.M0
    X = lambda a,t: (s0*a - M0)*exp(-riskFree.rate*t)
    listX=house.hood.sim.listX
    resolution=house.hood.sim.resolution
    
    listX = [X(i[0],i[1]) for i in listX]
    dist  = getDistribution(listX, resolution)
    dist.lastTime = t

    house.dist=dist
    return dist


def getDistribution(listX, resolution):
# Given a list of I.I.D. random numbers, returns the
# approximate probability distribution in the form of a python
# function.
    listX.sort()
    numData = int(1./resolution)
    CD = range( numData)
    multiplier = int(len(listX)*resolution)

    # The list is sorted, so the first 10 entries are in the bottom
    # percentile, the next 10 in the 2nd percentile, etc.
    for i in range(1,numData):
        CD[i] = listX[multiplier*i - 1]
    
    def dist(p):
    # Given a probability p, returns the approximate
    # 100p% value of the distribution.
    #
    # If p>.99 or p<.01 (for resolution=0.01), this function
    # simply returns dist(.99) or dist(.01) as appropriate.
    
        if p>1 or p<0:
            raise ValueError, "p is not a probability! You fool!"

        # If we have data with sufficient resolution, use it.
        p = (p/dist.resolution)
        
        if p<1:
            return CD[1]
        elif p>len(CD)-1:
            return CD[-1]
        elif p.is_integer():
            return CD[int(p)]
    
        # Otherwise, take a weighted average of the endpoints
        # of the appropriate interval.
        else:
            c = int( ceil( p) )
            f = int( floor(p) )
            return CD[c]*(p-f) + \
                   CD[f]*(c-p)
    
    # Attach the mean, variance, log-mean, and log-variance
    # just in case they're needed.
    mv = mean_and_variance(listX)
    dist.mean = mv['mean']
    dist.variance = mv['variance']

    mv = mean_and_variance([log(max(i,10**(-8))) for i in listX])
    dist.logMu = mv['mean']
    dist.logSigma = mv['variance']**.5

    dist.resolution = resolution

    return dist
    
##        
##t=0   
##N = neighborhoodClass(**nbhdDict)
##N.dist = getHoodDistribution(N)
##H=houseClass(N)
##H.value= N.medianPrice
##H.M0 = N.medianPrice*1.1
##E=expectation
##    

#########################################
##   The main simulation procedure
##########################################

step = dataClass(step=longTimeStep)
recursionDepth = dataClass(n=0)#db

for realization in range(realizations):
    print "Realization: ",realization
    if output.isOn: prep_outFile()
    t = 0

    index.last={}
    houses=set()

    if sim==None: t= duration + 1
    
    else:
        # Make neighborhoods
        neighborhoods=set()
        for d in nbhdDicts:
            # Homes on the market right away?
            n = d['initialHomes']#n = d.pop('initialHomes')

            nbhd = neighborhoodClass(**d)
            nbhd.dist = getHoodDistribution(nbhd)
            nbhd.sim = getHoodSimulation(nbhd)
            neighborhoods.add(nbhd)
            newHouses=set(houseClass(nbhd, t) for i in range(n))
            if trackOne:
                theHouse=newHouses.copy().pop()
                    
        # make investors
        investors = set(investorClass() for i in range(numInvestors))

        # Initialize containers
        auctions = list()
        primaries = list()
        secondaries=list()
        heps=set()
    
    while t<=duration:
        output.data['Time']=float(t)*360

        
        # If we're at the start of a new month,
        # reset time step size to 1 month.
        months = (t/longTimeStep)
        if months.denominator==1:#abs(months - round(months))<shortTimeStep:
            step.step = longTimeStep
            #print "Month", months#round(months)

        
        # Check for houses sold
        if HomeSalesOn:
            toRemove = set()
            for house in houses:
                if t>= house.sellDate:
                    toRemove.add(house)
            for house in toRemove:
                house.remove()
            toRemove.clear()

        #Check for new houses
        for hood in neighborhoods:
            
            newHouses.update(houseClass(hood,t) \
                            for i in range(hood.newHouses(t)))
            houses.update(newHouses)
            
            #Create new HEPS
            for house in newHouses:
                n = howManyHEPS(house)
                newHEPs = set(HEPclass(house) for i in range(n))
                heps.update(newHEPs)

            newHouses.clear()

        # Update neighborhood appreciation and home values.
        # Notice that this applies to any newly-added homes
        # as well; i.e., the "initial value" of a home is
        # in fact the value as of neighborhood.lastTime.
        for nbhd in neighborhoods:
            nbhd.appreciate(t)

        # For now, the list of investors is fixed.
        # Later:
        # newInvestors = getNewInvestors()
        # do something.

        #Do investors know that the homeowner is planning to sell?
        # to do: make this mean something
        for house in houses:
            if not house.knownSell:
                months = (house.sellDate - t)/12.
                p = 1-(months-1)**2 / 25.
                if random() < p:
                    house.knownSell = True

        # See if investors want to sell
        index.test=False
        for investor in investors:
            forSale = investor.getSellList(t)
            
            for hep in forSale:
                hep.needsSecondary = True

        # Update investor wealth
        for investor in investors:
            investor.update(t)
        
        # Make new auctions
        newAuctions()

        # Get investor buying power.
        bidderCash = getBidderCash()

        # Run Auctions
        recursionDepth.n=0#db
        

        theAuctions=set(auctions)
        runAuctions(theAuctions, bidderCash)
        
        #print "depth",recursionDepth.n
        
        # Finish auctions for the day; if there are any still
        # ongoing, then we need the short timestep.
        finishAuctions()

        t += step.step
        if trackOne:
            output.data.update({
                'Home Value':theHouse.value,
                # 'Waffles',    # If you don't need a particular item,
                                # just comment out the line -- DON'T DELETE!
                'Equity':theHouse.value-theHouse.M0,
                'HEP Expectation':valu(theHouse)
                })
        if output.isOn:makeOutput()             
        else: print output.data

if output.isOn:
        
    # Join up all the csv files
    longFile = '_'+output.name+'_long.csv'
    #broadFile = '_'+output.name+'_broad.csv'

    with open(longFile, 'wb') as L:
        writer = dictWriter(L, fieldNames, extrasaction='ignore')
        for r in range(realizations):
            r=str(r)
            with open('realization '+r+'.csv','rb') as f:
                reader = dictReader(f, fieldNames)
                for row in reader:
                    writer.writerow(row)
                d={}; d.setdefault('')
                writer.writerow(d); writer.writerow(d)
                
    #with open(broadFile, 'wb') as B:
        
                    
        
    



##########################################
#   Some testing and debugging stuff
##########################################

#meanGeo,
 #      sdGeo
 #               medianPrice,
   #              percent99,
    #             pHIP,
     #            pDisaster):
        
#meanGeo=1.063




\end{lstlisting}
